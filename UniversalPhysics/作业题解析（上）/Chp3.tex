\documentclass[a4paper,fleqn,twocolumn]{ctexart}
\usepackage{geometry}
\geometry{left=2.0cm,right=2.0cm,top=2.5cm,bottom=2.5cm}
\usepackage{amsmath}
\usepackage{amssymb}
\usepackage{graphicx}
\newcommand{\di}[1]{\mathrm{d}#1}
\newcommand{\p}[2]{\frac{\partial #1}{\partial #2}}
\newcommand{\pp}[2]{\frac{\partial ^2 #1}{\partial #2 ^2}}
\newcommand{\dy}[2]{\frac{\di{#1}}{\di{#2}}}
\newcommand{\ddy}[2]{\frac{\mathrm{d} ^2 #1}{\mathrm{d} #2 ^2}}

\begin{document}
\begin{section}{第三章}
\begin{subsection}{选择题}
1.A

小球受到重力和弹簧弹力做的功,动能不守恒。小球受到的合外力不为0,因此动量不守恒。

2.D

两船在过程中受到了人的摩擦力的作用,合外力不为零,动量不守恒。

3.C

碰撞前后两球动量守恒,因此总动量为0,说明碰撞前两球动量大小相同,方向相反。

4.D

两球碰撞后一起运动,说明为完全非弹性碰撞,存在机械能损失,因此机械能不守恒。两球还受到了弹簧弹力作用,合外力不为零,动量不守恒。

5.D

运动半周前后小球的速度大小不变,方向相反,因此冲量大小\(I=m\Delta v=2mv=2mr \omega\)。

6.C

A人向右跳落入B船后,由动量守恒,A船具有向左的速度,A人具有向右的速度,落入B船后使得A人,B人,B船都具有了向右的速度。B人再跳回A船后,由动量守恒,B人获得的动量朝左,A人和B船获得的动量向右,因此最终B船的速度向右,\(v_B>0\)。B人和A船的动量都向左,因此最终B人落入A船后,A船速度向左,\(v_A<0\)。

7.A

由公式\(E=\frac{p^2}{2m}\),\(p_1=\sqrt{2mE}\),\(p_2=\sqrt{2\times 4m\times E}=2\sqrt{2mE}\)(此处\(p_1\),\(p_2\)均为大小)。由于两个质点面对面运动,\(p_1\)和\(p_2\)符号相反,因此系统动量大小为\(p=p_2-p_1=\sqrt{2mE}\)。

8.A

小球转一周后速度大小和方向都与转前相同,因此动量增量为0,A对。由冲量定义式\(I=F\Delta t\),\(F\)和\(\Delta t\)均不为0,因此\(I\)也不为0,B错,同理可知C错误。在转动过程中小球速度的方向发生了改变,因此过程中动量不守恒,D错。

9.C

小球下落直到与板碰撞之前,对小球分析,有\(v=\sqrt{2gh}\),对板和弹簧分析,设此时弹簧的压缩量为\(x_0\),则由胡克定律有\(x_0=\frac{Mg}{k}\);此后对球与板的碰撞过程,有动量守恒\(mv=(m+M)v_0\),解得,\(v_0=\frac{m}{m+M}\sqrt{2gh}\);此后,小球与板一起运动,设弹簧最大压缩量为\(x\),由能量守恒,\(\frac{1}{2}(m+M)v_0^2+(M+m)gx+\frac{1}{2}kx^2=\frac{1}{2}k(x+x_0)^2\),代入数据解得,\(x=\frac{mg}{k}(1+\sqrt{1+\frac{2kh}{(M+m)g}})\),所以选择C项。

10.A

两球碰撞后以相同速度运动,则为完全非弹性碰撞,必有机械能损失,与完全弹性碰撞矛盾,A错误;若两球碰撞前速度大小相同,方向相反,则碰撞后速度互换,B正确;完全弹性碰撞的定义即为机械能守恒的碰撞,C正确;碰撞过程中动量守恒,D正确。
\end{subsection}
\begin{subsection}{填空题}
11. $10m/s$

以人和船为系统,过程中所受外力为$ 0 $,因此动量守恒。有
\begin{equation*}
(m+M)v=m(v'+\frac{v}{2})+M\frac{v}{2} 
\end{equation*}
代入数据解得$ v'=10m/s $

12. $ m\sqrt{2gh} $ \hspace{4em} 竖直向下

$(1)$由$ I=\Delta p=F\Delta t $,小球下落时间为$\Delta t=\sqrt{\frac{2h}{g}}$,下落过程所受合外力为$F=mg$,故动量增量为
\begin{equation*}
\Delta p=mg\cdot\sqrt{\frac{2h}{g}}=m\sqrt{2gh}
\end{equation*}

$(2)$由合外力(重力)方向竖直向下,可得动量增量方向竖直向下。

13.$\frac{5}{3}N\cdot s$,方向与力的方向同向 \hspace{2em} $\frac{5}{6}N$

由冲量定义,$I = \int F \di t$,对于本题来说,$F$是时间的函数,所以有
\begin{equation*}
I = \int_0^1 2t \di t + \int_1^2 2(2-t)^2 \di t = \frac{5}{3}N \cdot s
\end{equation*}
平均冲力为
\begin{equation*}
\overline{F} = \frac{I}{t} = \frac{5}{6} N
\end{equation*}

14.$-m\vec{v}$

对全过程,由动量定理,$\vec{I} = \Delta \vec{p} = -m\vec{v}$。

15.$2350kg\cdot m/s$ \hspace{2em} 与跑弹飞行方向相反

由动量定理,$I = \Delta p = m(v-v_0)$,代入数据得$I=2350kg\cdot m/s$,因为炮弹为减速运动,所以冲量方向与炮弹飞行方向相反。

16.$\frac{m+M}{m}\sqrt{2 \mu gs}$

设子弹刚发射时的速度为$v_0$,子弹射入木块后共速速度为$v$。对子弹射入木块的过程,有动量守恒$mv_0=(M+m)v$,此后木块和子弹共同运动,由动能定理
\begin{equation*}
\frac{1}{2}(m+M)v^2=\mu (m+M)gs
\end{equation*}
代入数据解得
\begin{equation*}
v_0=\frac{m+M}{m}\sqrt{2 \mu gs}
\end{equation*}

17.$\sqrt{\frac{kMl_0^2}{(M+m)^2}}$ \hspace{4em} $\sqrt{\frac{M}{M+nm}}l_0$

分析:每次油滴落入容器等价于油滴与容器发生了一次完全非弹性碰撞,该过程满足动量守恒;此后直到下一滴油滴落入容器之前,整体运动满足能量守恒。

求解:当弹簧第一次达到原长即容器第一次到达滴管下方(第一滴油滴未滴入)时,设容器速度为$v_0$,由能量守恒
\begin{equation*}
\frac{1}{2}Mv_0^2=\frac{1}{2}kl_0^2
\end{equation*}
可得
\begin{equation*}
v_0=\sqrt{\frac{k}{M}}l_0
\end{equation*}
再设第$n$滴油滴滴入容器后容器与油滴整体的速度为$v_n$;由动量守恒$P_0=P_n$,其中初状态动量$P_0=Mv_0$,第$n$滴油滴滴入后动量$P_n=(M+nm)v_n$,解得
\begin{equation*}
v_n=\frac{M}{M+nm}v_0
\end{equation*}
此后容器达到偏离$O$点的最大运动距离$x_n$过程中,由能量守恒,有
\begin{equation*}
\frac{1}{2}(M+nm)v_n^2=\frac{1}{2}kx_n^2
\end{equation*}
成立,从中可解出
\begin{equation*}
x_n=\sqrt{\frac{M+nm}{k}}v_n
\end{equation*}

故当$n=1$时,
\begin{equation*}
v_1=\sqrt{\frac{kMl_0^2}{(M+m)^2}}
\end{equation*}
滴入$n$滴油滴后,容器偏离$O$点的最大距离为
\begin{equation*}
x_n=\sqrt{\frac{M}{M+nm}}l_0
\end{equation*}

18.$m(\alpha\sin(\omega t)\vec{i}-(\beta\cos(\omega t)-\beta)\vec{j})$

由已知条件$a_x=\omega\alpha\cos(\omega t),a_y=\omega\beta\sin(\omega t)$,所以$v_x=\int a_x\di t=\alpha\sin(\omega t)+C_1,v_y=\int a_y\di t=-\beta\cos(\omega t)+C_2$,又由初始条件$t=0$时,$vec{v}=0$,可得$C_1=0,C_2=\beta$,所以质点在任一时刻的动量$\vec{p}(t)=m(\alpha\sin(\omega t)\vec{i}-(\beta\cos(\omega t)-\beta)\vec{j})$。

19.$\frac{mv}{\Delta t}$

由平均力计算公式可得$\overline{F}=\frac{I}{\Delta t}=\frac{mv}{\Delta t}$。

20.$\frac{1}{6}m(2v_B-v_A)^2$

因为发生完全弹性碰撞,结合能量守恒,可得碰撞过程中系统动能最小的时刻为弹性势能最大的时刻,亦即A、B共速的时刻,设共速时速度为$v$,则设B运动的方向为正方向由动量守恒有$2mv_B-mv_A=3mv$,此时的动能为
\begin{equation*}
E_k=\frac{1}{2}\cdot 3mv^2=\frac{1}{6}m(2v_B-v_A)^2
\end{equation*}
\end{subsection}
\begin{subsection}{计算题}
21.分析:$(1)$动量定理$(2)$因为$\overline{F}>>G$,所以可以忽略重力,再由平均力公式可以求解。

解:$(1)$由动量定理
\begin{equation*}
I=\Delta(mv)=m(v_2-v_1)=1.0\times(10+25)=35kg\cdot m/s
\end{equation*}
方向竖直向上

$(2)$设球对地面的平均冲力为$F$,地面对球的平均冲力为$F_0$。

对小球撞击地面的过程分析:

因为$\overline{F_0}>>G$,所以可以忽略重力\\
由平均力计算公式得
\begin{equation*}
\overline{F_0}=\frac{I}{t}=\frac{35}{0.02}N=1750N
\end{equation*}
再由牛顿第三定律可得
\begin{equation*}
\overline{F}=\overline{F_0}=1750N
\end{equation*}
方向竖直向下。

22.分析:全过程可分为两部分:一是子弹嵌入木块过程,动量守恒,二是子弹木块整体沿斜面上滑,能量守恒

解:子弹嵌入木块瞬间动量守恒,有
\begin{equation*}
mv=(m+M)v`
\end{equation*}
再由运动分解,木块速度可分解为垂直于斜面的速度和沿斜面方向的速度,因为木块此后被约束在斜面上运动,所以垂直于斜面的速度突变为$0$,即木块的运动速度为
\begin{equation*}
V=v'\cos\theta
\end{equation*}
对于木块和子弹的沿斜面上滑过程,有能量守恒
\begin{equation*}
\frac{1}{2}(m+M)V^2=(M+m)gl\sin\theta
\end{equation*}
联立上述三式可解得:
\begin{equation*}
V=\frac{mv\cos\theta}{M+m}\\
l=\frac{m^2v^2\cos^2\theta}{2(M+m)^2g\sin\theta}
\end{equation*}

23.分析:对于碰撞问题可以从能量守恒动量守恒两方面考虑

证明:设运动的小球为球1,质量为$m_1$,静止的小球为球2,质量为$m_2$,碰撞前球1的速度为$\vec{v}_0$,碰撞后球1、球2的速度分别为$\vec{v}_1、\vec{v}_2$,且有$\vec{v}_1$垂直于$\vec{v}_2$。

对碰撞过程:\\能量守恒
\begin{equation*}
\frac{1}{2}m_1v_0^2=\frac{1}{2}m_1v_1^2+\frac{1}{2}m_2v_2^2
\end{equation*}
即
\begin{equation}
v_0^2=v_1^2+\frac{m_2}{m_1}v_2^2
\end{equation}
动量守恒
\begin{equation*}
m_1\vec{v}_0=m_1\vec{v}_1+m_2\vec{v}_2
\end{equation*}
因为$\vec{v}_1$垂直于$\vec{v}_2$
所以由勾股定理可得
\begin{equation*}
m_1^2 v_1^2 + m_2^2 v_2^2 = m_1^2 v_0^2
\end{equation*}
\begin{equation}
v_0^2=v_1^2+\frac{m_2^2}{m_1^2}v_2^2
\end{equation}
$(2)-(1)$得
\begin{equation*}
\frac{m_2}{m_1}(\frac{m_2}{m_1}-1)v_2^2=0
\end{equation*}
因为$v_2^2 \ne 0$且$m_2 \ne 0$,所以
\begin{equation*}
\frac{m_2}{m_1}-1=0
\end{equation*}
即$m_1=m_2$

得证。

24.分析:将尘埃与飞船共同作为研究对象,则系统动量守恒,从而得到通过$m$与$v$的关系,再通过$\di m$的表达式与前式联立可以消去$\di m$再积分得到$v$和$t$的关系

解:设$t$时刻飞船的质量为$m$,速度为$v$。由初末状态系统动量守恒
\begin{equation*}
m_0 v_0=mv
\end{equation*}
即
\begin{equation*}
m=\frac{m_0 v_0}{v}
\end{equation*}
对上式取微分,可得
\begin{equation}
\di m=- \frac{m_0 v_0}{v^2} \di v
\end{equation}
又有
\begin{equation}
\di m=\rho sv\di t
\end{equation}
$(3)(4)$联立消去$\di m$再求积分,可得
\begin{equation*}
-\int_{v_0}^v \frac{\di v}{v^3}=\int_0^t \frac{\rho s}{m_0 v_0} \di t
\end{equation*}

解得
\begin{equation*}
v=\sqrt{\frac{m_0}{\rho sv_0 t+m_0}v_0}
\end{equation*}
\end{subsection}
\end{section}
\end{document}