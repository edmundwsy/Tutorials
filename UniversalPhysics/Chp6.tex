\documentclass[b5paper,opensource]{./template/qyxf-book}

% 这里可以自定义一些命令
\newcommand{\di}[1]{\mathrm{d}#1}
\newcommand{\p}[2]{\frac{\partial #1}{\partial #2}}
\newcommand{\pp}[2]{\frac{\partial ^2 #1}{\partial #2 ^2}}
\newcommand{\dy}[2]{\frac{\di{#1}}{\di{#2}}}
\newcommand{\ddy}[2]{\frac{\mathrm{d} ^2 #1}{\mathrm{d} #2 ^2}}
\newcommand{\zbj}[4]
{
	\draw (0,0) node[below left] {$ O $};
	\draw [->] (#1,0) -- (#2,0) node[right] {$ x $};
	\draw [->] (0,#3) -- (0,#4) node[right] {$ y $};
}


\begin{document}

\chapter{静电场1}
\section{选择题}

\exercise{1}C

\solve
两极板间场强为$E=\dfrac{q}{\varepsilon_0S}$,但这是由两极板共同贡献而成。
研究两板间作用力时,一个为施力物体,一个为受力物体。
单个无穷大带电平板的场强为$E_0=\dfrac{q}{2\varepsilon_0S}$,即在板的同侧是匀强电场,故作用力为$F=E_0\cdot q$

\exercise{2}D

\solve
取以O为圆心、$r,r+\Delta r$为内、外径的圆环面元,其上各点到通过圆心、垂直盘面的轴线上的任一点距离相等。
则:
\begin{gather*}
	\Delta S=2\pi r\di{r}\\
	\sigma=\dfrac{q}{\pi R^2}
\end{gather*}
故
\begin{equation*}
	\di{q}=\sigma \Delta S=\dfrac{2q}{R^2}r\di{r}
\end{equation*}

\exercise{3}C

\solve
计算面积分即可。在这里$x=x_0$,$\vec{E}$与$S$垂直。
\begin{equation*}
	 \iint\limits_S \vec{E}\di{S}=bx_0\cdot S=x_0bS
\end{equation*}

\exercise{4}A

\solve
由高斯定理,$E=\dfrac{q}{4\pi\varepsilon_0 r^2}$,只与$q$有关。

\exercise{5}D

\solve
由高斯定理,通过闭合曲面$S$的电通量只与面内的电荷有关,故电通量不变;由电场强度的叠加原理知,$q$在P点产生的场强不变,但Q的改变。由于是矢量加法,场强一定会改变。

\exercise{6}C

\solve
由高斯定理:
\begin{align*}
	E\cdot 4\pi r^2&=\dfrac{1}{\varepsilon_0}\int_0^r\rho\cdot 4\pi r^2\di{r}\\
	&=\dfrac{4\pi b}{\varepsilon_0}\int_0^r e^{-kr}\di{r}\\
	&=\dfrac{4\pi b}{\varepsilon_0 k}(1-e^{-kr})\\
\end{align*}
所以
\begin{gather*}
	E=\dfrac{b}{\varepsilon_0 k}\dfrac{1-e^{-kr}}{r^2}\\
	r=\dfrac{1}{k}\text{时,}E_1=\dfrac{b}{\varepsilon_0}\cdot k(1-e^{-1})\\
	r=\dfrac{2}{k}\text{时,}E_2=\dfrac{b}{\varepsilon_0}\cdot \dfrac{k}{4}(1-e^{-2})
\end{gather*}
故
\begin{equation*}
	E_2=\dfrac{1-e^{-2}}{4(1-e^{-1})}E_1\approx 0.34E_1
\end{equation*}

\exercise{7}D

\solve
简单定积分。以棒上远离场点的一端点为原点,指向场点的方向为x轴正方向。
\begin{align*}
	E&=\int_{0}^{L}\dfrac{1}{4\pi \varepsilon_0}\dfrac{\lambda\di{x}}{{(r+\frac{L}{2}-x)}^2}\\
	&=\dfrac{Q}{4\pi \varepsilon_0 L}\cdot \dfrac{1}{{r+\frac{L}{2}-x}}\bigg|_0^L\\
	&=\dfrac{Q}{\pi \varepsilon_0(4r^2-L^2)}
\end{align*}

\exercise{8}B

\solve
由高斯定理知,$\sum q_{\text{内}}=0$,即净电荷为零,但不能说明高斯面$S$内任一点都没有电荷。

\exercise{9}D

\solve
题中所谓的“点”是一个固定的点,只不过刚开始在气球外,最后在气球内。如图1。注意气球是有厚度的!
\begin{figure}[!h]
	\centering
	\includegraphics[width=\textwidth]{Chp6_illus1.ai}
	\caption{练习9\quad 气球膨胀}
\end{figure}
点在气球外时,球外场强不变;点在球壳中时,高斯面包围的电荷减小,场强减小;点在气球内时,场强为0。

\exercise{10}B

\solve
该系统可以看作是半径为$R_1$、电荷体密度为$\rho$的均匀带电球体和半径为$R_2$、电荷体密度为$-\rho$的均匀带电球体的组合。应用均匀带电球体内部电场强度的矢量式
\footnote{
	$E=
	\begin{cases}
	\dfrac{\rho}{3\varepsilon_0}\vec{r} & r\leqslant R,\\
	\dfrac{\rho R^3}{3\varepsilon_0 r^3}\vec{r} & r > R
	\end{cases}$

	标量式中,$r>R$时,$E=\dfrac{\rho R^3}{3\varepsilon_0 r^2}$。
}得:
\begin{align*}
	E&=\dfrac{\rho}{3\varepsilon_0}\vec{r}_1+\dfrac{-\rho}{3\varepsilon_0}\vec{r}_2\\
	&\text{($r_1,r_2$分别是$O_1,O_2$到场点的矢径)}\\
	&=\dfrac{\rho}{3\varepsilon_0}(\vec{r}_1-\vec{r}_2)\\
	&=\dfrac{\rho}{3\varepsilon_0}\overrightarrow{O_1O_2}
\end{align*}
即空腔内为匀强电场。

\section{填空题}

\exercise{11}$-\dfrac{2VR}{3r^2}$\quad 沿球心到该点的矢径向外

\solve
由电势的定义和均匀带电球体电场强度公式(见练习10脚注)得:
\[
0-V=\int_{0}^{R}\dfrac{\rho r}{3\varepsilon_0}\di{r}+\int_{R}^{+\infty}\dfrac{\rho R^3}{3\varepsilon_0 r^2}\di{r}
\]
整理得:
\[
V=-\dfrac{\rho R^2}{2\varepsilon_0}
\]
与场强公式对比即知:
\[
E=-\dfrac{2VR}{3r^2}
\]
由于$V<0$,可知球体带正电荷,故场强方向沿球心到该点的矢径向外。

\exercise{12}$\dfrac{\lambda_1\lambda_2}{2\pi \varepsilon_0}\ln\dfrac{a+l}{a}$\quad 沿有限长带电线指向无限长带电线(或在纸面上向左)

\solve
以两直线交点为原点(或有限长带电线左端点),远离无限长直线为x轴正方向,沿有限长带电线建立坐标系。

由无限长带电直线场强公式得:
\begin{align*}
	F&=\int_{a}^{a+l}\dfrac{\lambda_1}{2\pi \varepsilon_0 x}\lambda_2\di{x}\\
	&=\dfrac{\lambda_1\lambda_2}{2\pi \varepsilon_0}\ln\dfrac{a+l}{a}
\end{align*}
两线均带正电,故有限长带电线受力方向为x轴正向。注意题目问的是无限长带电线的受力,而且答题时不要引入“x轴”。

\exercise{13}$-\dfrac{\sigma a^2}{2\varepsilon_0}$\quad$\dfrac{\sigma a^2}{2\varepsilon_0}$\quad 0

\solve
情景参见练习3、练习5。注意此处考虑曲面的外侧为正向,内侧为负。

\exercise{14}0\quad $\dfrac{(Q+4a\lambda)}{\varepsilon_0}$

\solve
第一空由对称性易得。第二空使用高斯定理,$\sum q_{\text{内}}=Q+4a\cdot\lambda$。注意题目问的不是场强。

\exercise{15}$\dfrac{\sqrt{2}a}{2}$

\solve
如图2,设$q_0$与$A$的连线和$AB$的夹角为$\theta$。由对称性,合力方向与$AB$平行。
\begin{figure}[!h]
	\centering
	\includegraphics[width=0.5\textwidth]{Chp6_illus2.ai}
	\caption{练习15\quad 受力分析}
\end{figure}
\begin{align*}
	E&=2\cdot \dfrac{1}{4\pi\varepsilon_0}\dfrac{qq_0}{r^2+a^2}\sin\theta\\
	&=\dfrac{qq_0}{2\pi\varepsilon_0}\dfrac{r}{{(r^2+a^2)}^{\frac{3}{2}}}
\end{align*}
令$f(r)=\dfrac{r}{{(r^2+a^2)}^{\frac{3}{2}}}$,则$f'(r)=\dfrac{a^2-2r^2}{{(r^2+a^2)}^{\frac{5}{2}}}$。
由导数知识得:$a^2-2r^2=0\text{即}r=\dfrac{\sqrt{2}}{2}a$时,$f(r)$取得极大值。

或者用三角换元。令$r=a\tan\theta$,则
\begin{gather*}
	f(\theta)=\dfrac{1}{a^2}\sin\theta\cos^2\theta\\
	f'(\theta)=\dfrac{1}{a^2}\cos\theta(\cos^2\theta-2\sin^2\theta)
\end{gather*}
由导数知识得:$\tan\theta=\dfrac{\sqrt{2}}{2}\text{即}r=\dfrac{\sqrt{2}}{2}a$时,$f(\theta)$取得极大值。

\exercise{16}$\dfrac{\sigma^2\Delta S}{2\varepsilon_0}$

\solve
均匀带电球面在其表面处产生的电场强度为
\[
E=\dfrac{\sigma\cdot 4\pi R^2}{4\pi\varepsilon_0R^2}=\dfrac{\sigma}{\varepsilon_0}\\
\]
但该电场也是由所取的面元和剩余部分共同贡献而成(同练习1),故作如下分析:

\begin{figure}[!h]
	\centering
	\includegraphics[width=0.5\textwidth]{Chp6_illus3.ai}
	\caption{练习16\quad 场强分布}
\end{figure}
如图3,红色部分为$\Delta S$,我们分球壳内外来研究。剩余部分球壳在$\Delta S$内、外侧(无限接近$\Delta S$且对称的)两点P、Q处产生的场强均可看作是$E_0$,而$\Delta S$在P、Q处产生的场强$E_1=E_2$。那么由题意知:
\begin{gather*}
	\text{(P点)}E_0-E_1=0\\
	\text{(Q点)}E_0+E_1=E\\
\end{gather*}
解得:
\[
E_0=E_1=\dfrac{1}{2}E=\dfrac{\sigma}{2\varepsilon_0}
\]
那么
\begin{gather*}
	q_S=\sigma\Delta S\\
	F=E_0q_S=\dfrac{\sigma^2\Delta S}{2\varepsilon_0}
\end{gather*}

\exercise{17}$-\dfrac{\lambda}{2\pi\varepsilon_0R}\vec{j}$

\solve
\[
\di{E}=\dfrac{\lambda \di{s}}{4\pi \varepsilon_0 R^2}=\dfrac{\lambda \di{\theta}}{4\pi \varepsilon_0 R}
\]
注意到电场强度是由圆环指向O点,故:
\[
\di{E_x}=\di{E}\cdot (-\cos\theta),\di{E_y}=\di{E}\cdot (-\sin\theta)
\]
积分\footnote{此后因对称性而场强为零的情景不再积分计算}得:
\begin{gather*}
	E_x=\dfrac{\lambda}{4\pi \varepsilon_0 R}\int_0^{\pi} -\cos\theta\di{\theta}=0\\
	E_y=\dfrac{\lambda}{4\pi \varepsilon_0 R}\int_0^{\pi} -\sin\theta\di{\theta}=-\dfrac{\lambda}{2\pi\varepsilon_0R}
\end{gather*}
故得$E=-\dfrac{\lambda}{2\pi\varepsilon_0R}\vec{j}$

\exercise{18}$\dfrac{\vec{p}}{2\pi\varepsilon_0x^3}$

\solve
设电偶极子的两电荷电量为$q$,距离为$l$。
\begin{align*}
	E&=\dfrac{q}{4\pi\varepsilon_0}\cdot \dfrac{1}{(x-\frac{l}{2})^2}+\dfrac{-q}{4\pi\varepsilon_0}\cdot \dfrac{1}{(x+\frac{l}{2})^2}\\
	&=\dfrac{q}{2\pi\varepsilon_0}\dfrac{xl}{{(x^2-\frac{l^2}{4})}^2}
\end{align*}
$x\gg l$时,$x^2-\dfrac{l^2}{4}\approx x^2$。同时考虑方向,可以得到:
\begin{align*}
	E&=\dfrac{q\vec{l}}{2\pi\varepsilon_0x^3}\\
	&=\dfrac{\vec{p}}{2\pi\varepsilon_0x^3}
\end{align*}

\exercise{19}0

\solve
复习高斯定理。

\exercise{20}$\dfrac{\lambda d}{\pi\varepsilon_0 a^2}$\quad 从O点指向细线端点连线中点

\solve
使用补偿法即可。而且$d\ll a$,故将空缺处看作点电荷。
\[
E=\dfrac{\lambda d}{4\pi\varepsilon_0{(\frac{a}{2})}^2}=\dfrac{\lambda d}{\pi\varepsilon_0 a^2}
\]
空缺了一些产生向下场强的电荷,故场强方向向上。

\section{解答题}

\exercise{21}

\solve
以O为极点,建立如图3所示的坐标系。极轴穿过圆弧的终点。
\begin{figure}[!h]
	\centering
	\includegraphics[width=0.5\textwidth]{Chp6_illus4.ai}
	\caption{练习21\quad 坐标系}
\end{figure}
把柱面分成无限根无限长直导线(沿母线方向),设无限长导线长度为$l$,则其线密度$\lambda=\dfrac{\di{q}}{l}=\dfrac{\di{(\sigma l\cdot R\theta)}}{l}=\sigma R\di{\theta}$

则
\begin{gather*}
	\di{E}=\dfrac{\lambda}{2\pi\varepsilon_0R}=\dfrac{\sigma}{2\pi\varepsilon_0}\di{\theta}\\
	\di{E_x}=-\di{E}\cos\theta,\di{E_y}=-\di{E}\sin\theta\\
	E_x=\dfrac{\sigma}{2\pi\varepsilon_0}\int_{-\frac{\theta}{2}}^{\frac{\theta}{2}}-\cos\theta\di{\theta}=-\dfrac{\sigma\sin\frac{\theta}{2}}{\pi\varepsilon_0}\\
	E_y=\dfrac{\sigma}{2\pi\varepsilon_0}\int_{-\frac{\theta}{2}}^{\frac{\theta}{2}}-\sin\theta\di{\theta}=0
\end{gather*}
故所求场强大小为$\dfrac{\sigma\sin\frac{\theta}{2}}{\pi\varepsilon_0}$

\exercise{22}

\solve
由对称性,场强方向沿x轴。取面积微元(见练习2),则:
\begin{align*}
	E=&\int_0^R\dfrac{1}{4\pi\varepsilon_0}\cdot\dfrac{\frac{\sigma_0r}{R}\cdot 2\pi r\di{r}}{r^2+x^2}\cdot\dfrac{x}{\sqrt{r^2+x^2}}\\
	&=\dfrac{\sigma_0x}{2\varepsilon_0R}\int_0^R\dfrac{r^2\di{r}}{{(r^2+x^2)}^\frac{3}{2}}\\
	&=\dfrac{\sigma_0x}{2\varepsilon_0R}[-\dfrac{r}{\sqrt{r^2+x^2}}+\ln(r+\sqrt{r^2+x^2})]\Big|_0^R\\
	&=\dfrac{\sigma_0x}{2\varepsilon_0R}\big(\ln\dfrac{R+\sqrt{R^2+x^2}}{x}-\dfrac{R}{\sqrt{R^2+x^2}}\big)
\end{align*}
方向沿x轴正向。

\exercise{23}

\solve
由题,取半径为$r(R_1<r<R_2)$的球面为高斯面。由高斯定理:
\begin{equation*}
	E\cdot 4\pi r^2=\dfrac{1}{\varepsilon_0}(Q+\int_{R_1}^r\dfrac{A}{r}\cdot 4\pi r^2\di{r})
\end{equation*}
计算得:
\begin{equation*}
	E=\dfrac{A}{2\varepsilon_0}+\dfrac{Q-2\pi AR_1^2}{4\pi r^2\varepsilon_0}
\end{equation*}
题目中$E$与$r$无关,故:
\begin{gather*}
	Q-2\pi AR_1^2=0\\
	A=\dfrac{Q}{2\pi R_1^2}
\end{gather*}

\exercise{24}

\solve
由对称性,场强方向沿z轴。
以垂直于z轴的平面去截球壳得到小圆,取该圆为长、$R\di{\theta}$为宽的面元。注意代入$\sigma_0$。
\begin{align*}
	\di{E}&=\dfrac{\sigma_0\cdot 2\pi R\sin \theta R\di{\theta}}{4\pi \varepsilon_0 R^2}\\
	&=\dfrac{\sigma}{2\varepsilon_0}\sin\theta\cos\theta\di{\theta}\\
	E_z&=\int_0^{\frac{\pi}{2}}\di{E}\cdot(-\cos\theta)\di{\theta}\\
	&=-\dfrac{\sigma_0}{2\varepsilon_0}\int_0^{\frac{\pi}{2}}\sin\theta\cos^2\theta\di{\theta}\\
	&=-\dfrac{\sigma_0}{6\varepsilon_0}
\end{align*}
$\therefore$所求电场强度为$\dfrac{\sigma_0}{6\varepsilon_0}$,方向沿z轴负方向。

\end{document}
