%!TEX program = xelatex
\documentclass[a4paper,fleqn,twocolumn]{article}
\usepackage{xeCJK}
\usepackage{geometry}
\geometry{left=2.0cm,right=2.0cm,top=2.5cm,bottom=2.5cm}
\usepackage{amsmath}
\usepackage{amssymb}
\usepackage{graphics}
\setCJKmainfont{SimSun}
\newcommand{\di}[1]{\mathrm{d}#1}
\newcommand{\p}[2]{\frac{\partial #1}{\partial #2}}
\newcommand{\pp}[2]{\frac{\partial ^2 #1}{\partial #2 ^2}}
\newcommand{\dy}[2]{\frac{\di{#1}}{\di{#2}}}
\newcommand{\ddy}[2]{\frac{\mathrm{d} ^2 #1}{\mathrm{d} #2 ^2}}

\begin{document}
\section*{第五章}
    1.\par
        由书P153知,$Re_{crit}=2100$
        \begin{align*}
            v_{柴crit}  &=\frac{Re\upsilon}{D}=\frac{2100\times4.41\times10^{-6}}{0.1524}\\
                        &=0.0608\left(m/s\right)\\
            v_{水crit}  &=\frac{Re\upsilon}{D}=\frac{2100\times1.13\times10^{-6}}{0.1524}\\
                        &=0.0156\left(m/s\right)
        \end{align*}
    2.
        \begin{align*}
            &Re_水=\frac{VD}{\upsilon}=\frac{1.607\times0.305}{1.13\times10^{-6}}=2.88\times10^5\gg2100\\
            &\therefore\text{为湍流}\\
            &Re_柴=\frac{VD}{\upsilon}=\frac{1.607\times0.305}{205\times10^{-6}}=1.59\times10^5<2100\\
            &\therefore\text{为层流}\\
        \end{align*}
    3.
        \begin{align*}
            &Re_{crit}=2100\\
            &V=\frac{Q}{S}=\frac{Q}{\frac{\pi}{4}D^2}=\frac{4Q}{\pi D^2}\\
            &\text{代入:}\\
            &Re=\frac{VD}{\upsilon}=\frac{4QD}{\pi D^2\upsilon}=\frac{4Q}{\pi D\upsilon}\\
            &\therefore D_{min}=\frac{4Q}{Re\pi\upsilon}=\frac{4\times5.67\times10^{-3}}{2100\pi\times6.08\times10^{-6}}=0.565(m)\\
            &\therefore D\geqslant 0.565m
        \end{align*}
    4.\par
        公式见书P332
        \begin{align*}
            \therefore Le  &=0.06Re_D\times D\\
                            &=0.06\times1500\times0.01\\
                            &=0.9(m)
        \end{align*}
    5.
        \[V=\frac{4Q}{\pi D^2}\text{(推导见第3题)}\]
        \begin{align*}
            \therefore Re  &=\frac{VD}{\upsilon}=\frac{4Q}{\pi D\upsilon}\\
                            &=\frac{4\times0.8\times10^{-3}}{\pi\times0.1\times\upsilon}=\frac{0.01019}{\upsilon}
        \end{align*}
        \begin{gather*}
            \text{当}Re>2100\text{时,为湍流}\\
            \text{此时:}\upsilon<\frac{0.01019}{Re}=\frac{0.01019}{2100}=4.85\times10^{-6}\\
            \text{查表(书P534-535):}\upsilon_空=1.51\times10^{-5}(m^2/s)\\
            \upsilon_水=1.004\times10^{-6}(m^2/s)\\
            \upsilon_{甘油}=\frac{1.49}{1260}=1.18\times10^{-3}(m^2/s)
        \end{gather*}
        $\therefore$(1)(2)(5)(6)为层流,(3)(4)为湍流。\\
    6.\par
        (此题默认球直径为0.1m)
        \begin{align*}
            \text{空气:}&V_{crit}=\frac{\upsilon Re}{D}&=\frac{1.51\times10^{-5}\times250000}{0.1}=37.75(m/s)\\
            \text{水:}&V_{crit}&=\frac{1.004\times10^{-6}\times250000}{0.1}=2.51(m/s)\\
            \text{氢气:}&V_{crit}&=\frac{\frac{0.9\times10^{-5}}{0.0839}\times250000}{0.1}=268.2(m/s)
        \end{align*}
    7.
        \begin{gather*}
            \because\nabla\cdot V=\p{u}{x}+\p{v}{y}+ \p{w}{z}=0.\therefore\text{为不可压流动}\\
            \tau_{xx}=2\mu \p{u}{x}=0.\text{同理}\tau_{yy}=\tau_{zz}=0\\
            \tau_{xy}=\mu\left( \p{u}{y}+ \p{v}{x}\right)=\mu(zt+zt)=2\mu zt=0.02zt\\
            \tau_{xz}=\mu\left( \p{u}{z}+ \p{w}{x}\right)=\mu yt=0.01yt\\
            \tau_{yz}=\mu\left( \p{v}{z}+ \p{w}{y}\right)=\mu xt=0.01xt
        \end{gather*}
    8.
    \begin{gather*}
        \because\nabla\cdot V=0.\therefore\text{为不可压流动}\\
        \tau_{xx}=\tau_{yy}=\tau_{zz}=0\\
        \tau_{xy}=\mu\left( \p{u}{y}+ \p{v}{x}\right)=\mu(2+1)=0.024{Pa\cdot s}\\
        \tau_{xz}=\mu\left( \p{u}{z}+ \p{w}{x}\right)=\mu(3+2)=0.040{Pa\cdot s}\\
        \tau_{yz}=\mu\left( \p{v}{z}+ \p{w}{y}\right)=\mu(3+4)=0.056{Pa\cdot s}
    \end{gather*}
    9.
    \begin{gather*}
        \because\text{层流}.\therefore v=0\text{(分层流动,y方向速度为0)}\\
        \because\nabla\cdot V=0.\therefore\text{为不可压流动}
    \end{gather*}
    \begin{align*}
        \therefore f_{sx}  &=\mu\nabla^2u=\mu\pp{u}{y}\\
                            &=\mu u_{max}\left[-2\left(\frac{1}{n}\right)\frac{1}{n}\right]=-\frac{2\mu u_{max}}{h^2}\\
                            &=-3.76(N/m^3)    
    \end{align*}
    	10.11.12三题大体思路一致,将速度场代入N-S方程,得到关于$p^*$的方程。以$p^*$有解为条件,可得到关于n或r的微分方程,求解即得到结果。由于过程过于复杂,笔者认为理解思路即可,此处不再赘述。\\
    13.
        \begin{gather*}
            \text{已知条件:}v=0,\p{p^*}{x}=0.\\
            \text{由连续方程:}\nabla\cdot V=\p{u}{x}+0=0,\therefore \p{u}{x}=0.\\
            x\text{轴方向的$N-S$方程:}0=-\pp{u}{y}\\
            \therefore u=Ay+B.\text{代入}u\left.\right|_{y=0}=U_1,u\left.\right|_{y=b}=U_2,\\
            \therefore u=\frac{u_2-u_1}{b}y+u_1
        \end{gather*}
    14.\par 
    (1)由书P143页,有:
    \begin{gather}
    	\ddy{u}{y}=\frac{1}{\mu}\dy{p^*}{x},\quad \therefore \dy{u}{y}=\frac{1}{\mu}\dy{p^*}{x}y+c_1.\notag\\
    	\text{由书P149:}\dy{u}{y}\left.\right|_{y=h}=0\notag\\
    	\therefore c_1=-\frac{h}{\mu}\dy{p^*}{x},\quad \dy{u}{y}=\frac{1}{\mu}\dy{p^*}{x}(y-h)\notag\\
    	\text{再做积分:}u=\frac{y}{\mu}\dy{p^*}{x}(\frac{y}{2}-h)\\
    	\text{而}\vec{g}=g(\sin\theta\vec{i}-\cos\theta\vec{j})\notag\\
    	\therefore p^*=p+\rho g(\cos\theta y-\sin\theta x)\notag\\
    	\because\text{上表面为自由液面}\therefore \dy{p}{x}=0\text{(见书P148)}\notag\\
    	\therefore \dy{p^*}{x}=-\rho g\sin\theta\notag\\
    	\text{代入(1),得}u=\frac{\rho g}{\mu}y\left(h-\frac{y}{2}\right)\sin\theta\notag
	\end{gather}    
	(2)
	\begin{align*}
		\dot{Q}	&=\int_{y=0}^{y=h}u\mathrm{d}S=\int_{y=0}^{y=h}u(\mathrm{d}y\times 1)\text{(单位宽度为1)}\\
				&=\int_{0}^{h}\frac{\rho g}{\mu}\sin\theta y\left(h-\frac{y}{2}\right)\mathrm{d}y\\
				&=\frac{\rho g\sin\theta}{\mu}\left(\frac{1}{2}hy^2-\frac{1}{6}y^3\right)\left.\right|_0^h\\
				&=\frac{\rho g\sin\theta h^3}{3\mu}
	\end{align*}
	(3)
	\begin{gather*}
		\bar{u}=\frac{\dot{Q}}{S}=\frac{\dot{Q}}{h\times 1}=\frac{\rho g\sin\theta h^2}{3\mu}\\
		\text{代入速度分布式,得:}y\left(h-\frac{y}{2}\right)=\frac{h^2}{3}\\
		\text{解得:}y=\frac{3-\sqrt{3}}{3}\text{(另一根超过h,舍去)}
	\end{gather*}
	15.
	\begin{gather*}
		\text{由z轴方向N-S方程:}\\
		0=-\p{p}{z}+g\sin30^\circ+\upsilon\frac{1}{r}\frac{\di{}}{\di{r}}\left(r\dy{V_z}{r}\right)\\
		\text{由于压强为常数,}\p{p}{z}=0\quad \therefore -\frac{g}{2\upsilon}=\frac{1}{r}\frac{\di{}}{\di{r}}\left(r\dy{V_z}{r}\right)\\
		\text{解得:}V_z=-\frac{gr^2}{8\upsilon}+c_1\ln r+c_2\\
		\text{代入边界条件:}V=\frac{\rho g}{8\mu}(r_0^2-r^2)
	\end{gather*}
	\begin{align*}
		\therefore \dot{Q}=SV	&=\int_{0}^{r_0}\frac{\rho g}{8\mu}(r_0^2-r^2)\times 2\pi r\di{r}\\
								&=\frac{\rho g}{8\mu}\pi\left(r_0^2r^2-\frac{1}{2}r^4\right)\\
								&=\frac{\rho g\pi}{16\mu}r_0^4
	\end{align*}
	16.\par 
	以$r=37.5mm+0.025mm$处为$y=0$面,$r=37.5mm$处为$y=h$面,问题化为平面库埃特流动。\\
	\begin{gather*}
		U=\frac{100}{60}\pi d=\frac{100\pi\times 0.075}{60}=\frac{\pi}{8}\\
		\tau_w=-\mu\frac{U}{h}=-0.2\times\frac{pi}{8\times0.025\times10^{-3}}=-1000\pi\\
		\text{而}S=\pi dh=\pi\times0.075\times0.16=0.03770(m^2)\\
		M=F\times r=\tau Sr=-1000\pi\times0.03770\times\frac{0.075}{2}=-4.44(N\cdot m)
	\end{gather*}
	17.\par 
	参考书P151下方对圆柱体流体微元的分析:
	\begin{gather}
		\tau=\frac{r}{2}\frac{\Delta p^*}{L}=K\left(\dy{u}{y}\right)^n\\
		\text{重力在水平管道流动方向分量为0,}\therefore \Delta p^*=\Delta p\notag\\
		\text{由几何意义,}|dy|=|dr|\therefore |\dy{u}{r}|=|\dy{u}{y}|\notag\\
		\text{考虑到r增大时,u变小,}\dy{u}{r}<0\notag\\
		\text{由(2):}\dy{u}{r}=\pm\sqrt[n]{\frac{r\Delta p}{2KL}}\text{因此应取负号}\notag\\
		\text{两边积分,}u=-\sqrt[n]{\frac{\Delta p}{2KL}}\frac{n}{n+1}r^{\frac{n+1}{n}}+c\notag\\
		\text{由边界条件}r=R,u=0\notag\\
		\therefore u=\sqrt[n]{\frac{\Delta p}{2KL}}\frac{n}{n+1}\left[R^{\frac{n+1}{n}}-r^{\frac{n+1}{n}}\right]\notag\\
		\text{即}=\sqrt[n]{\frac{R\Delta p}{2KL}}\frac{nR}{n+1}\left[1-\frac{r}{R}^{\frac{n+1}{n}}\right]\notag
	\end{gather}
	18.\par 
	可参考书P153例5.6
	\begin{gather*}
		V_z=\frac{1}{4\mu}\dy{p^*}{z}r^2+c_1\ln r+c_2\\
		\text{由于压强为常数,重力在水平管道流动方向分量为0,}\p{p^*}{z}=\p{p}{z}=0\\
		\text{代入边界条件}r=r_0,V=0;r=R,V=V_0:\\
		\text{解得:}V=\frac{V_0\ln\frac{r}{r_0}}{\ln\frac{R}{r_0}}\\
		\tau_w=-\mu\dy{V}{r}=\frac{\mu V_0}{\ln\frac{R}{r_0}}\frac{\frac{1}{r_0}}{\frac{r}{r_0}}\\
		=\frac{\mu V_0}{R\ln\frac{R}{r_0}}\\
		\therefore F=\tau S=\frac{\mu V_0}{R\ln\frac{R}{r_0}}\times(2\pi R\times1)=\frac{2\pi\mu V_0}{\ln\frac{R}{r_0}}
	\end{gather*}
\end{document}