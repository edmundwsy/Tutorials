\section{数学课介绍}
首先说一下总体上的建议, 就是要多读书,多做题。并且在这两者之间要有一个平衡。具体地说,多读书一定要建立在读懂的情况下,如果没完全读懂的话,读再多书作用也不大。所以读书一定要做题。做题的量根据自己水平而定,如果课后题一下子就能看出答案那基本上不用做题了,可以读进一步的书了。如果课后题很多不会做的话,那就一定要把例题和课后习题好好做一遍。最后达到课后题基本都会做的水平,就可以读下一本书了。\par
另外,如果书里面不理解的内容非常多,那就是基础没有打扎实,建议自己检查一下哪些基础知识没学好,然后回去补(基本上数分和线代没学好就寸步难行了,这两门是最重要的基础,希望一定要重视,不然补起来就比较难了)。\par
最后,一定要重视例子。因为越是抽象的东西,有例子做支撑才能理解好。对于每个定理,都需要知道它应用的几个具体例子,那样的话至少可以达到一个初步的理解。\par
在掌握课内基础知识的内容后,很多同学会开始迷茫,不知道是学一些别的课程,还是继续做更困难的习题;是读一本书,还是找个老师做做科研训练。首先我不建议急于做很多问题,因为这个世界上有太多太多未解决的问题,而数学的目标也不是解决所有的问题。数学追求的是理解而不仅仅是答案,所以趁大学低年级间,时间比较充裕,应该更自由的接触一下数学的各个领域,而不是被困在科研问题里,限制了自己的兴趣和想法。至于接触数学的哪些方面,以什么方式接触,其实不那么重要。如果你已经发现了一个或几个使你感兴趣的领域,不妨直接想办法了解,读书,问老师,如果碰巧有志同道合的同学可以一起办个讨论班都是很好的方法;如果你还没发现,或者发现了却无从下手,我建议可以抽时间看一些数学史,最经典的就是M.Klein的《古今数学思想》,当你了解了历史,不但可以大大帮助你理解一切所学的数学课程,而且还会让你萌生对数学浓烈的情感,使你成为一个有趣的数学学习者。\par
(以下推荐的参考书都是不错的书,但是希望同学们不仅仅是想到的时候拿出来翻一下,而是要花心思好好读。如果我列了很多参考书,你没必要每本都读,但是如果要读的话就花心思读好,克服一切困难,而不是随便翻翻,结果哪本书都没明白)
\subsection{数学史}
这一块我们也不是专家,我们学校也没有专家,所以我就只推荐久负盛名的M.Klein的《古今数学思想》,还有很多像《普林斯顿数学指南》等类型的书籍中也有很多历史内容,和一些数学的典范应用,都值得有兴趣的同学们一读。了解数学的故事,了解数学家的故事,是我们大学教育缺失的重要环节,如果你学了所有的必修课,却给亲友讲不出任何数学家的故事,谈不出任何数学的趣事,学数学专业可能把你学的很无趣,但我想这些书不仅可以改变这个现状,还可以作为生活的调味剂,倒也不一定要从头读到尾,有空读几节,坚持下去!
\subsection{数学分析}
数学分析是大学里面最重要的一门课,也是学分最高的一门课,是所有分析课的基础。对此一定要重视。我建议是不仅要把书后面的题好好做一做,而且最好还能看一些课外的参考书。推荐:菲赫金哥尔茨《微积分学教程》(这本书很厚,看一部分自己需要的就行),Rudin《数学分析原理》,谢惠民《数学分析习题课讲义》(没必要全做,但是尽力做一下会有收获的)除此之外,Zorich 的《数学分析》框架比较大,把数学分析的很多内容推广到更一般的情况下来做,可以时常参考,或者等学完数学分析后进行阅读来让分析的理解上升一个台阶。数学分析曲线曲面积分与微分几何相关,可以了解外微分与stokes定理,推荐\,Munkres\,《流形上的分析》
微分学可以在简单了解赋范线性空间的基础上尝试了解无穷维赋范线性空间的微分学,这个不会比多元微分学困难,但学习后对理解会大有裨益。
\subsection{高等代数}
高等代数是大学代数里面最重要的课,也是所有代数课的基础。不管是在哪个领域都是很重要的工具。丘老师的书内容很多,习题也很多,我建议主要把他布置的习题做了,没必要所有题都做,不然的话会花太多时间。然后除了丘老师的教材,还可以找几本参考教材(北大蓝以中的高等代数教材是不错的参考),学习一下不同老师的风格,比较一下证明的异同,会理解更深入一点。\par
还有就是,这门课一定要勤于计算,并且证明要反复读。因为代数的证明可能会比数学分析的证明难懂一点,所以需要自己拿几个例子好好算一下,并且反复看证明。\par
另外,学有余力的同学还可以自己思考一些问题,比方说把线性代数的结果推广到除环或者主理想整环上会怎么样?


\subsection{数论}

数论有一定难度,但是初等数论里面很多内容是和近世代数紧密联系的。学的时候建议自己学一点简单的近世代数,两者结合可以理解更深刻。
比方说研究\,$\pmod{m}$\,就是在一个循环群里面研究,原根的存在性就是说的\,$(\mathbb{Z}/m\mathbb{Z})^*$\,是循环群。
比较难的是二次互反律,一般对证明不做什么考察要求,但是如果对此感兴趣的话,可以去看看\,Serre\,数论教程的第一章,利用有限域对这个的一个诠释,是很有趣的。另外,二次互反律更进一步的发展在代数数论里面,这个理论是比较难的,但是有兴趣的话可以在之后学学,需要一定的\,Galois\,理论和交换代数基础。

\subsection{常微分方程}

在计算方面好好算一些例子,在证明方面主要考察的还是数分,所以数分一定要学好。\par
如果想对常微的理论作更深的了解,可以阅读李承治老师的《常微分方程教程》,由于李老师本人是分支理论的大师,所以里面不仅对解的存在唯一性定理作了更细致的讲解,也包含了许多定性理论和分支理论的内容,这也是所谓“动力系统”中的内容。\par
如果对分支理论很感兴趣,可以进一步阅读李老师和我们学校马知恩、周义仓老师合著的《常微分方程定性与稳定性方法》,也是一本起点比较低、讲解比较透彻的比较亲民的好书。常微分方程有深刻的几何内涵,阿尔诺德 的《常微分方程》是很好的著作。更深入的内容内容可以在学完微分流形和复变函数后接触。

\subsection{近世代数}

近世代数同样也是是一门比较难的课,很多东西是全新的。建议自己要按着书上把基本性质好好推一遍。一开始遇到题不会做也很正常,不要害怕看答案,做不出来的习题当做例题,当例子足够多了之后,就会适应抽象的语言,遇到问题甚至会情不自禁的抽象化处理。考试时这门课对大家要求反倒是不高,参考课本和丘维声《近世代数》(有习题解答)即可。但如果是想对这一门课有更系统的了解的同学,我们推荐参考书:Rotman《群论导论》(前几章就够了),GTM167《域论和Galois理论》(证明很详细),冯克勤《近世代数引论》,推荐大家了解一下范畴论中最最基本的概念(Wiki百科上就够了),因为范畴论是集合论之后又一大语言革命,如果说集合论是把一切看成集合,那么范畴就是把一切看成所谓的“态射”,范畴的语言已经成为代数学与几何学的基本语言,它不仅可以让你更系统的思考抽象代数,更会给你的思维方式天翻地覆的变化。温馨提示:范畴论就像集合论一样,必须了解,但如果太深入,反倒是有些浪费时间。最后提及一些李文威编写的《代数学方法》,是新出的好书,但观点很高难度很大,对代数有兴趣的同学可以拿来参考。

\subsection{数值分析}
主要考察线性代数和数学分析的知识,但其中很多内容的思想在代数,组合,数论,泛函分析都会遇到,所以大家应该对这门相对轻松的课程抱有认真的态度。
\subsection{拓扑学}

我们教材(用的是尤承业老师的书)题量不是太大,而且有提示,建议基本上做一遍。点集拓扑部分推荐\,Munkres\,的拓扑学。代数拓强烈推荐A.Hatcher的《algebraic topology》第零章的内容,学习了CW复形后,你会发现你的几何世界变得截然不同,除此之外,Armstrong的《基础拓扑学》里面有很多课本中内容不同的讲法,图文并茂,引人入胜。温馨提醒:几何的灵魂在于生动的例子,不要困在复杂的定义中,当你真正用心去学习上面所提到的几何的内容,你会发现儿时一切的想象在此完整的实现。\par


\subsection{实变函数}

推荐接触抽象测度论,抽象测度论的方法和传统实分析截然不同,用测度思考,可以把很多问题变得简单明快,真正理解现代分析的力量。抽象测度论也是学习概率论的基础,在学习抽象测度论后,概率论的学习将变得势如破竹。书籍推荐:胡适耕《实变函数》,Cohn《Measure Theory》(很厚,用于参考,未来从事概率或者统计方向的同学,可以抽假期细致的学习一遍),Folland《Real Analysis》(观点很高,不妨在实变函数课后期和泛函分析课程学习中拿来参考)


\subsection{复变函数}
这门课非常好,可以学到很多有用而且有趣的东西。希望大家能喜欢上这门课。教材是\,Stein\,的复分析,里面有很多理论和应用,题目质量也很高,老师推荐的参考书是方企勤《复变函数》写的更接近中国风格,也和考试更贴近,所以建议大家多参考这本书,但注意前三章的内容不是重点却并不简单,一开始可以简单翻阅,不要恋战。另外,龚昇的复分析也不错,是一本好的参考书(不过也不简单)。大家学习的时候要注意里面的几何理论和PDE的理论。这门课是黎曼面和PDE的很好的预备课。
\subsection{概率论}
与实变关系密切,但是一般本科课程不会从完全严格的实变的基础开始讲,而是更加注重应用,但有兴趣的同学一定要在很好学习抽象测度论后严格化书上所有的证明,会让理解再上一层。难点是极限定理和一部分比较难的概率的计算。中心极限定理和大数定律的证明建议好好看一下,这个非常重要,并且主要还是数分知识。前面的古典概型会涉及到一些组合知识,虽然技巧性很强但要求并没有那么高,好好做一下题应该能应付。
