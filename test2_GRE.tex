\documentclass[blue, pc, cn]{elegantnote}

\title{GRE备考指南V1.0}
\author{钱学森84费立涵}
\institute{小破楼资料编写小组\LaTeX}
\version{1.0}
\date{2019 . 3 . 16}
\begin{document}
    \maketitle
    \newpage
    {\centering
    更新内容:
    \begin{enumerate}
        \item[1]增加了对“背一遍”的详细描述
        \item[2]增加了AW部分评分标准
        \item[3]更新了最后的个人感受
        \item[4]修改一些小错误
    \end{enumerate}
    }
    \newpage
    \begin{flushleft}
        {\LARGE写在前面:} \\
        {\small
        微臣有线下课和线上课。 \\
        线下课只有北京有,14天全天,16000左右。 \\
        线上课主力课程是
        \begin{itemize}
            \item 填空  \textit{方法论+1200题中200题};
            \item 阅读  \textit{方法论+200篇中30多篇};
            \item 数学  \textit{知识点回顾+精选200题};
            \item 写作onepass   \textit{方法论+现场写作,一般一个1000左右};
            \item 安神班    \textit{9.9,考前3个小时讲点东西。这些课程每个月一次}。
        \end{itemize}
      
        偶尔还会有一些特殊课程,如救命800词,讲词汇;填空刷1200题,阅读刷文章的课程。
        这些爆品课一般较便宜,500左右。
    
    
        \paragraph{}备考GRE首选当然是去北京上线下课,效率最高,上完之后自己再刷点题就可以上考场了。
        对于分数要求不高的同学,直接听线上onepass课程就可以,课上认真听老师讲,onepass的训练是足够的。上完onepass再自己刷很多题就ok。
        但对于分数要求高的同学(325/330+),请从基础开始(如本文所述),稳扎稳打,你会收获不仅仅是GRE的高分。
    
        另外,看课的过程中,也不妨思考思考学习学习这几位老师的人生哲学。
        
        当然了,本文仅供参考。很多时候,自己闯出一些路,自己吃了亏,才是收获最大的。
        \textbf{所以也不要只按本文所述,要探索适合自己的方式,会有意想不到的收获。}
        }
    \end{flushleft} 
   
    \newpage
    \tableofcontents
    \newpage
    
    \section[1]{书籍介绍}
        微臣每一本书的正文之前的所有部分都要仔细看,包括前言,序言,使用说明等等。

        彩虹书全套几乎每一本都有用的,所以要买还是一起买吧,还便宜。

        \begin{enumerate}
            \item[1]\textbf{救命800}\\唯一特殊的那本。背单词初期资料。包含了GRE,托福和四六级三个模块,每个模块800个词。包含中英文解释和近义词。
            \item[2]\textbf{小紫(小3000)}\\是大3000的压缩版,体积小,容易携带。只保留了单词的中英文解释,排序是从靠后字母排序的。
                    如coda在coea的前面。如果没有移动背书需求其实就不需要这本了。微臣资料里有该书的excel词表。
            \item[3]\textbf{小红书(短语搭配)}\\包含了365个短语。附有练习。想要冲高分的话此书必背过。
            \item[4] \textbf{绿皮书(助记精炼)}\\为大3000配套书籍。助记部分讲解如何用词根词缀记住大3000的单词,精炼部分为大3000内单词的配套练习,有词义连线,句子填空,找同义词三部分。
                    其他微臣学生对该书评价极高,称其大大加快了背单词的速度,同时词根词缀也记得更牢。
                    然而就我个人经验来看,此书没什么用。反感词根词缀记忆法的同学不必买此书。微臣资料里有该书电子版。
            \item[5]\textbf{ 大3000(核心词汇考法精析)}\\包含3000个高频词汇,包含中英文解释,例句,同反义词,是从开始备考到考前一天每天都要翻很长时间的书。
                    5遍以内纯裸考。微臣资料里有该书的excel词表和电子版。
            \item[6]\textbf{ 24套(橘皮书)}\\本书为解析,微臣资料里有电子版题目。SAT填空题目合集,由于SAT考察的单词和GRE几乎一致,只不过逻辑上很简单。可以说,只要认识单词就能做对,
            因此是极佳的巩固GRE单词的资料。题目经过修改和词汇调整,使得符合GRE的题目形式(六选二,双空)。
            \item[7]\textbf{黄皮书(大黄)(长难句300句)}\\300个长难句+分析+中文翻译,还配有50个句子的视频讲解。
            攻克长难句超级利器,读文章和paper的好办法。微臣资料里有该书老版本的电子版。

            \item[8]\textbf{36套(蓝皮书)}\\本书为解析,微臣资料里有电子版题目。老GRE的填空题目合集,都是五选一(单空题),因此不管是词汇考法还是逻辑考法,都与新G有所不同,因此参考价值相对于金皮书就没有那么大了。可用于24套到1200题的过渡。如果没有时间或者24套做的很扎实,则建议跳过这一步或者不用做完,就直接去做金皮书。然而从经验来看,大部分人还是需要这个过渡的。
            \item[9]\textbf{(阅读)白皮书}\\本书为解析,微臣资料里有电子版题目。老GRE的阅读大白本中的文章,204篇。老G和新G阅读考察内容几乎一样,但形式有所不同,因此相较于金皮书和200篇,白皮书的价值更小一些。只建议有余力的时候再做。
            \item[10]\textbf{黑皮书(写作)}\\包含写作指导,OG分析和20篇issue,20篇argument的范文中英文全篇和点评。写作必备参考书。
            \item[11]\textbf{金皮书(语文高频题目精析)}\\本书为解析,微臣资料里有电子版题目。GWC(1200题和200篇)中经典和很高频的题目(200+40),比1200+200要更有价值。
            \item[12]\textbf{备考胜经(粉皮书)}\\为微臣留美公众号的精选推送文章合集。都是常考的重点,闲暇时光值得一看,同时包含了部分留学内容。
        \end{enumerate}

    \subsection{资料介绍}
        \paragraph{填空1200题,阅读200篇}
        市面上流传有2013-2016年的机经题目,微臣经过校对和修改之后的资料(然而还是有地方出错。)。GWC是这两个东西的总称。金皮书上的题目便来源于这两份资料。
        \footnote{GRE题库较为特殊,ETS会不断往里加新题,但老题不会被撤出,只是抽取概率会小一点。
        而且每次考试的每道题,都是从题库里随机抽取,并不是以section为单位抽取题目,所以做1200题和200篇之后,
        在考场上会碰到几道原题。但千万不要抱着撞原题的想法去狂刷1200+200,没有解析没有反馈的练习是没有意义的。}

        \paragraph{Onepass讲义}
        onepass课程结束后会发放onepass纸质讲义。填空为1200题纸质版,阅读为200篇纸质版,写作为onepass写作讲义,包含30篇argument范文(大部分含有中文翻译)及10+篇issue范文(每次onepass课上的现场写作范文会收录进来),数学为500题讲义。

        \paragraph{微臣资料}
        关注微臣留美公众号,右下角里有彩虹书籍,点击下载。

    \subsection{视频地址}	万门大学搜索“GRE”,会弹出来一堆关于陈琦的视频。找到你需要的即可。
                    \underline{在再要你命3000小程序里找也可以。}

\end{document}